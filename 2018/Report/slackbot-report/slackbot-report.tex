\documentclass[12pt]{jsarticle}
\usepackage[dvipdfmx]{graphicx}
\textheight = 25truecm
\textwidth = 18truecm
\topmargin = -1.5truecm
\oddsidemargin = -1truecm
\evensidemargin = -1truecm
\marginparwidth = -1truecm

\def\theenumii{\Alph{enumii}}
\def\theenumiii{\alph{enumiii}}
\def\labelenumi{(\theenumi)}
\def\labelenumiii{(\theenumiii)}
\def\theenumiv{\roman{enumiv}}
\def\labelenumiv{(\theenumiv)}
\usepackage{comment}
\usepackage{url}

%%%%%%%%%%%%%%%%%%%%%%%%%%%%%%%%%%%%%%%%%%%%%%%%%%%%%%%%%%%%%%%%
%% sty/ にある研究室独自のスタイルファイル
\usepackage{jtygm}  % フォントに関する余計な警告を消す
\usepackage{nutils} % insertfigure, figref, tabref マクロ

\def\figdir{./figs} % 図のディレクトリ
\def\figext{pdf}    % 図のファイルの拡張子

\begin{document}
%%%%%%%%%%%%%%%%%%%%%%%%%%%%
%% 表題
%%%%%%%%%%%%%%%%%%%%%%%%%%%%
\begin{center}
{\LARGE SlackBot プログラム作成の報告書}
\end{center}

\begin{flushright}
  2018/4/25\\
  藤原 裕貴
\end{flushright}
%%%%%%%%%%%%%%%%%%%%%%%%%%%%
%% 概要
%%%%%%%%%%%%%%%%%%%%%%%%%%%%
\section{概要}
\label{sec:introduction}
本資料は2018年度B4新人研修課題の報告書である.新人研修課題として SlackBot プログラムを作成した.
Slack\cite{Slack}とは Web 上で利用できるチームコミュニケーションツールである.SlackBot とはある契機により自動でSlackに発言するプログラムのことである.
本資料では,課題内容,理解できなかった部分,作成できなかった機能,および自主的に作成した機能について述べる.


\section{課題内容}
課題として, SlackBot プログラムを Ruby で作成する.具体的には以下の2つを行う.

\begin{enumerate}
\item 任意の文字列を発言するプログラムの作成

Slack でユーザが``「(任意の文字列)」と言って"を含む発言をした場合に SlackBot が``(任意の文字列)"を発言するプログラムを作成する.

\item SlackBot プログラムへの機能追加

Slack 以外の Web サービスの API や Webhook を利用した機能を追加する.たとえば,ユーザの発言を契機に SlackBot が雨の情報を発言する機能である.
\end{enumerate}

本課題で使用する Ruby のバージョンは2.5.1である.

\section{理解できなかった部分}
理解できなかった部分を以下に示す.
\begin{enumerate}
\item \verb|initialize| メソッドに記述されている以下のコードの動作
  \begin{verbatim}
@incoming_webhook = ENV['INCOMING_WEBHOOK_URL'] || config["incoming_webhook_url"]
  \end{verbatim}
    このコードは,演算子\verb+||+で Heroku の環境変数と\verb|settings.yml|に記述された URL の論理和の計算をしている.
    これが必要な理由と動作が理解できなかった.

\end{enumerate}

\section{作成できなかった機能}
作成できなかった機能を以下に示す.
\begin{enumerate}
\item 指定した Outgoing WebHooks 以外からの POST を拒否する機能
\item 指定された地点間の経路を示した画像を表示する機能
\end{enumerate}

\section{自主的に作成した機能}
移動手段,出発地点,および到着地点から以下の情報を SlackBot が発言する機能を作成した.
\begin{enumerate}
\item 出発地点から到着地点までの距離
\item 出発地点から到着地点までの移動にかかる時間
\item 出発地点から到着地点までの経路の詳細を示した Google Map へのリンク
\end{enumerate}


\bibliographystyle{ipsjunsrt}
\bibliography{mybibdata}




\end{document}
